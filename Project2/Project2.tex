\documentclass[english,a4paper, 11pt]{article}
\usepackage[T1]{fontenc} %for å bruke æøå
\usepackage[utf8]{inputenc}
\usepackage{graphicx} %for å inkludere grafikk
\usepackage{verbatim} %for å inkludere filer med tegn LaTeX ikke liker
\usepackage{mathpazo}
\usepackage{hyperref}
\usepackage{float}
\usepackage{gensymb}
\usepackage{amsmath}

\title{FYS3150: Project 2}
\author{Henrik Haugerud Carlsen and Martin Moen Carstensen}
\date{\today}
\begin{document}

\maketitle

\begin{abstract}
\centering

\end{abstract}


\section{Introduction}

To show that the Jacobi mathod can be used it is important to show that an orthogonal(or unitary) transformation of a basis preserves the orthogonality trait of the basis.
Lets consider a basis. $\vec{v_i}$:

\begin{equation}
\mathbf{v}_i = \begin{bmatrix} v_{i1} \\ \dots \\ \dots \\v_{in} \end{bmatrix}
\label{eq1}
\end{equation}

The basis is assumed to be orthogonal, which implies the following trait: $ \vec{v_j}^T\vec{v_i} = \delta_{ij}$. Our task is not to show that a transformation $U\vec{v_i} = \vec{w_i}$ preserves the orthogonality, such that :

\begin{equation}
\vec{w_j}^T\vec{w_i} = \delta_{ij}\;,
\label{eq2}
\end{equation}

which equates to

\begin{equation}
(U\vec{v_j})^T U\vec{v_i} = \delta_{ij}\;.
\label{eq3}
\end{equation}

Since the transformation is considered orthogonal we have $(AB)^T = B^TA^T$ and $U^TU = UU^T = I$, giving us

\begin{equation}
\vec{v_j}^TU^TU\vec{v_i} = \vec{v_j}I\vec{v_i} = \delta_{ij}\;,
\label{eq4}
\end{equation}

returning us to the original expression for orthogonality of the dot product:

\begin{equation}
\vec{v_j}\vec{v_i} = \delta_{ij}\;.
\label{eq5}
\end{equation}

Thus we have shown that orthogonality is perserved under the transformation $U$.



\section{Method}


\section{Results}


\section{Discussion}




\section{Conclusion}



\section{References}
 






\end{document}

